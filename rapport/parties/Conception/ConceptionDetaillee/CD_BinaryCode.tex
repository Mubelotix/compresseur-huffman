\subsubsection{BinaryCode}

\begin{algorithme}
    \begin{enregistrement}{BinaryCode}
        \champEnregistrement{bits}{\tableauUneDimension{0..MAX}{de}{ Bit}}
        \champEnregistrement{length}{\naturel}
    \end{enregistrement}
\end{algorithme}

\begin{algorithme}
    \typeEnumere{Bit}{ZERO, ONE}
\end{algorithme}

\begin{algorithme}
    \fonction{binaryCode}{}{BinaryCode}{}{bc : BinaryCode}{
        \affecter{bc.length}{0}
        \affecter{bc.bits}{\tableauUneDimension{0..MAX}{de}{ Bit}}
        \retourner{bc}
    }
\end{algorithme}

\begin{algorithme}
    \procedure{addBit}{\paramEntreeSortie{bc : BinaryCode}, \paramEntree{b : Bit}}{}{}{
        \affecter{bc.bits[bc.length]}{b}
        \affecter{bc.length}{bc.length + 1}
    }
\end{algorithme}

\begin{algorithme}
    \fonction{getLength}{bc : BinaryCode}{\naturel}{}{}{
        \retourner{bc.length}}

\end{algorithme}

\begin{algorithme}
    \fonction{getBit}{bc : BinaryCode, position : \naturel}{Bit}{0 <= position < getLength(bc)}{}{
        \retourner{bc.bits[position]}
    }
\end{algorithme}

\begin{algorithme}
    \procedure{concatenate}{\paramEntreeSortie{bc1 : BinaryCode}, \paramEntree{bc2 : BinaryCode}}{}{}{
        \pour{i}{1}{getLength(bc2)}{}{
            \instruction{addBit(bc1, getBit(bc2, i))}
        }
    }
\end{algorithme}
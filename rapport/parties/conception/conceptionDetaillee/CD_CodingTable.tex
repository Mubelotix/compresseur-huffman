\subsubsection{CodingTable}
    
\constante{MAX}{256}

\begin{algorithme} 
\begin{enregistrement}{optionalBinaryCode}
    \champEnregistrement{present}{\booleen}
    \champEnregistrement{binaryCode}{BinaryCode}
\end{enregistrement}
\end{algorithme}
\vspace*{0.5cm}

\begin{algorithme} 
\begin{enregistrement}{CodingTable}
    \champEnregistrement{tab}{\tableau{1..MAX}{de}{optionalBinaryCode}}
\end{enregistrement}
\end{algorithme}
\vspace*{0.5cm}

\begin{algorithme} 
    \fonction{codingTable}{}{CodingTable}{}{table : CodingTable,i : \naturelNonNul}
    {\pour{i}{1}{MAX}{}{
        \instruction{\affecter{\champ{table}{tab[i].present}}{0}}
    }
    \retourner{table}}
\end{algorithme}
\vspace*{0.5cm}

\begin{algorithme} 
    \fonction{isEmpty}{table : CodingTable}{\booleen}{}{i : \naturelNonNul}
    {\pour{i}{1}{MAX}{}{
        \sialorssinon{\champ{table}{tab[i].present}}
        {\retourner{0}}{}
    }
    {\retourner{1}}}
\end{algorithme}
\vspace*{0.5cm}

\begin{algorithme} 
    \fonction{contains}{table : CodingTable, byte : \naturel}{\booleen}{}{}
    {\retourner{\champ{table}{tab[byte].present}}}
\end{algorithme}
\vspace*{0.5cm}

\begin{algorithme} 
    \fonction{getBinaryCode}{table : CodingTable, byte : \naturel}{BinaryCode}{contains(table,byte)}{}
    {\retourner{\champ{table}{tab[byte]}{.binaryCode}}}   
\end{algorithme}
\vspace*{0.5cm}

\begin{algorithme}
    \procedure{add}{\paramEntreeSortie{table: CodingTable} \paramEntree{byte: \naturel, code : BinaryCode} }{non contains(table,byte)}{}
    {\affecter{\champ{table}{tab[byte]}{.present}}{VRAI}
    \affecter{\champ{table}{tab[byte]}{.binaryCode}}{code}}
\end{algorithme}
\vspace*{0.5cm}

\begin{algorithme} 
    \fonction{getByte}{table : CodingTable, code : BinaryCode}{Byte}{}{i : \naturelNonNul}
    {\pour{i}{1}{MAX}{}{
        \sialorssinon{\champ{table}{tab[i].present} et \champ{table}{tab[i].binaryCode}=code} 
        {\retourner{byte(i et 0b80, i et 0b40, i et 0b20,i et 0b10,i et 0b08,i et 0b04,i et 0b02,i et 0b01)}}{}
    }
    {\retourner{byte(0,0,0,0,0,0,0,0)}}}
\end{algorithme}

\documentclass[12pt]{article}

\usepackage{packages/imports}
\usepackage{packages/pseudocode}
\usepackage{packages/renews}
\usepackage{packages/analyseDescendante}

\begin{document}

%page de garde
 \begin{titlepage}


    \centering{

    \vspace{20px}
    \textbf{Systèmes d'exploitation}
    
    \vspace{20px}
    
    \Huge
    Les commandes de base d'Unix
    
    \normalsize

    \vspace{165px} %40 de base avec le logo à 100
    %\includegraphics[width=100px{documents/PhD_Couverture_Fond.pdf}
    
    \vspace{120px}
    \textbf{Réalisé par}

    \vspace{5px}
    \normalsize
    BERAL Quentin\\
    SENTCHEV Vassili\\
    
    \normalsize %normalement en large
    \vspace{20px}

    \vspace{20px}

  \vspace{45px}
    Année universitaire 2023-2024

    \includegraphics[width=200 px]{img/INSA.jpg}
    
}

\end{titlepage}
\newpage

\section{Introduction}
Confrontés au défi de créer un outil de compression performant basé sur l'algorithme de Huffman, nous avons adopté une approche stratégique où chaque membre s'est vu confier des responsabilités spécifiques à chaque étape du cycle en V. Au cœur de notre méthodologie, une règle fondamentale prévalait : à l'étape n du cycle en V, aucun étudiant pouvait être responsable de la partie qu'il lui a été attribuée à l'étape n-1. Au travers de ce rapport, nous détaillerons chaque étape du cycle en V, mettant en lumière les contributions individuelles de chaque membre. \par
L'analyse a marqué le point de départ de notre aventure. Durant cette phase, nous nous sommes penchés sur la spécification des Types Abstraits de Données (TAD) et sur l'élaboration d'une analyse descendante pour saisir les exigences fondamentales de notre compresseur Huffman.\par
La conception préliminaire, suivant naturellement l'analyse, a constitué le premier pas concret vers la réalisation de notre compresseur. Chaque membre a été chargé de définir les signatures des fonctions et procédures utilisées dans le projet. \par
La conception détaillée, point central de notre démarche, s'est avérée être le lieu où les idées abstraites ont pris forme. Chaque membre, en se basant sur les spécifications de conception préliminaire, a élaboré les algorithmes détaillés des fonctions et procédures nécessaires à la réalisation du compresseur.
\newpage

%sommaire
\tableofcontents

\listoffigures

%parties

\newpage
\section{Analyse}
Au début du projet, notre équipe s'est attelée à définir les Types Abstraits de Données (TAD) nécessaires et à élaborer une analyse descendante. Cependant, chaque membre était dédié à une composante particulière de cette analyse. Vous trouverez la matrice de répartition des tâches en dernière page de ce rapport.\newline
\subsection{TADs}
\subsubsection{BinaryCode}

\begin{tad}
    \tadNom{BinaryCode}
    \tadDependances{Byte}
    \begin{tadOperations}{BinaryCode}
        \tadOperation{binaryCode}{}{BinaryCode}
        \tadOperation{addBit}{BinaryCode × Bit}{BinaryCode}
        \tadOperation{getLength}{BinaryCode}{\naturel}
        \tadOperationAvecPreconditions{getBit}{BinaryCode × \naturel}{Bit}
        \tadOperation{concatenate}{BinaryCode × BinaryCode}{BinaryCode}
        \tadOperationAvecPreconditions{removeFirstByte}{BinaryCode}{Byte}
        \tadOperation{appendByte}{BinaryCode × Byte}{BinaryCode}
        \tadOperation{prefix}{BinaryCode × \naturel}{BinaryCode}
        \tadOperation{suffix}{BinaryCode × \naturel}{BinaryCode}
        \tadOperation{equals}{BinaryCode × BinaryCode}{\booleen}
    \end{tadOperations}
    \begin{tadSemantiques}{BinaryCode}
        \tadSemantique{binaryCode}{Crée un code binaire.}
        \tadSemantique{addBit}{Ajoute un bit au code binaire.}
        \tadSemantique{getLength}{Renvoie la longueur du code binaire (le nombre de bits).}
        \tadSemantique{getBit}{Renvoie le bit à la position donnée dans le code binaire.}
        \tadSemantique{concatenate}{Concatène deux codes binaires.}
      \end{tadSemantiques}    
    \begin{tadAxiomes}
        \tadAxiome{getLength(concatenate(cb1,cb2))=getLength(cb1)+getLength(cb2)}
    \end{tadAxiomes}
    \begin{tadPreconditions}{getBit}
        \tadPrecondition{getBit(BinaryCode, position)}{0<=position<getLength(BinaryCode)}
        \end{tadPreconditions}
\end{tad}
\subsubsection{Byte}

\begin{tad}
    \tadNom{Byte}
    \tadDependances{\naturel, Bit}
    \begin{tadOperations}{byteToNatural}
        \tadOperation{octet}{\tadHuitParams{Bit}{Bit}{Bit}{Bit}{Bit}{Bit}{Bit}{Bit}}{\tadParams{Byte}}
        \tadOperationAvecPreconditions{getBit}{\tadDeuxParams{Byte}{\naturel}}{Bit}
        \tadOperationAvecPreconditions{setBit}{\tadTroisParams{Byte}{\naturel}{Bit}}{Byte}
        \tadOperationAvecPreconditions{byteToNatural}{Byte}{\naturel}
    \end{tadOperations}
    \begin{tadSemantiques}{charToByte}
        \tadSemantique{getBit}{Renvoie le bit associé à la position donnée.}
        \tadSemantique{setBit}{Fixe le bit associé à la position donnée.}
        \tadSemantique{byteToNatural}{Renvoie le naturel associé à l'octet entré.}
    \end{tadSemantiques}
    \begin{tadAxiomes}
        \tadAxiome{\forall i \in [0, 7], getBit(createByte(bit_0,bit_1,bit_2,bit_3,bit_4,bit_5,bit_6,bit_7), i)=bit_i}
    \end{tadAxiomes}
    \begin{tadPreconditions}{charToByte(c)}
        \tadPrecondition{getBit(byte,i)}{0 $\leq$ i $\leq$ 7}
        \tadPrecondition{byteToNatural(b)}{0 $\leq$ b $\leq$ 255}
    \end{tadPreconditions}
\end{tad}

\subsubsection{CodingTable}

\begin{tad}
    \tadNom{CodingTable}
    \tadDependances{HuffmanTree, \naturelNonNul, \caractere, Binary}

    \begin{tadOperations}{getBinaryCode}
        \tadOperation{codingTable}{}{CodingTable}
        \tadOperation{tableSize}{CodingTable}{\naturelNonNul}
        \tadOperationAvecPreconditions{getBinaryCode}{\tadDeuxParams{CodingTable}{Byte}}{BinaryCode}
        \tadOperation{getChar}{\tadDeuxParams{CodingTable}{BinaryCode}}{\caractere}
    \end{tadOperations}

    \begin{tadSemantiques}{getBinaryCode}
        \tadSemantique{codingTable}{Crée une table de codage vide.}
        \tadSemantique{tableSize}{Renvoie le nombre de charactères présents dans une table de codage.}
        \tadSemantique{getBinaryCode}{Renvoie le code binaire associé à un octet dans une table de codage.}
        \tadSemantique{getChar}{Renvoie le charactère associé à un code binaire dans une table de codage.}
    \end{tadSemantiques}

    \begin{tadPreconditions}{getBinaryCode(c, t)}
        \tadPrecondition{getBinaryCode(table, byte)}{code $\in$ table}
        \tadPrecondition{getChar(table, code)}{char $\in$ table}
    \end{tadPreconditions}
\end{tad}

\subsubsection{HuffmanTree}

\begin{tad}
    \tadNom{HuffmanTree}
    \tadDependances{Statistics, Byte}

    \begin{tadOperations}{HuffmanTree}
        \tadOperation{huffmanTree}{Statistics}{HuffmanTree}
        \tadOperationAvecPreconditions{getRightChild}{HuffmanTree}{HuffmanTree}
        \tadOperationAvecPreconditions{getLeftChild}{HuffmanTree}{HuffmanTree}
        \tadOperationAvecPreconditions{getByte}{HuffmanTree}{Byte}
        \tadOperation{getOccurence}{HuffmanTree}{\entier}
        \tadOperation{isALeaf}{HuffmanTree}{\booleen}
        \tadOperation{createLeaf}{\entier}{Byte}
        \tadOperation{createNode}{\tadDeuxParams{HuffmanTree}{HuffmanTree}}{HuffmanTree}
    \end{tadOperations}

    \begin{tadSemantiques}{getOccurence(HuffmanTree)}
        \tadSemantique{huffmanTree(Statistics)}{Crée un arbre de Huffman à partir des statistiques fournies.}
        \tadSemantique{getRightChild(HuffmanTree)}{Renvoie le fils droit de l'arbre.}
        \tadSemantique{getLeftChild(HuffmanTree)}{Renvoie le fils gauche de l'arbre.}
        \tadSemantique{getByte(HuffmanTree)}{Renvoie l'octet de l'arbre.}
        \tadSemantique{getOccurence(HuffmanTree)}{Renvoie l'occurence de l'octet de l'arbre.}
        \tadSemantique{isALeaf(HuffmanTree)}{Renvoie vrai si l'arbre est une feuille, faux sinon.}
        \tadSemantique{createLeaf(\entier)}{Crée un feille pour l'arbre d'huffman}
        \tadSemantique{createnode(HuffmanTree, HuffmanTree)}{Crée un arbre d'huffman à partir d'un fils droit et d'un fils gauche}
    \end{tadSemantiques}

    \begin{tadPreconditions}{getLeftChild(A)}
        \tadPrecondition{getByte(A)}{isALeaf(A)}
        \tadPrecondition{getLeftChild(A)}{non isALeaf(A)}
        \tadPrecondition{getRightChild(A)}{non isALeaf(A)}
    \end{tadPreconditions}
\end{tad}


\subsubsection{TAD PriorityQueue}

\begin{tad}
  \tadNom{PriorityQueue}
  \tadDependances{ArbreDeHuffman, Naturel}
  
  \begin{tadOperations}{PriorityQueueEmpty}
    \tadOperation{PriorityQueueEmpty}{}{PriorityQueue}
    \tadOperation{ajouterArbre}{ArbreDeHuffman X PriorityQueue}{PriorityQueue}
    \tadOperationAvecPreconditions{defilerArbre}{PriorityQueue}{Arbre}
    \tadOperation{sizeQueue}{PriorityQueue}{Naturel}
  \end{tadOperations}

  \begin{tadSemantiques}{PriorityQueueEmpty}
    \tadSemantique{PriorityQueueEmpty}{Crée une file de priorité Empty.}
    \tadSemantique{ajouterArbre}{Ajoute un arbre de Huffman à son emplacement respectif dans une file de priorité.}
    \tadSemantique{defilerArbre}{Défile l'arbre en tête d'une file de priorité.}
    \tadSemantique{sizeQueue}{Renvoie le nombre d'arbre de Huffman présents dans une file de priorité.}
  \end{tadSemantiques}

  \begin{tadAxiomes}
    \tadAxiome{ajouterArbre(defilerArbre(f),f) = f}
    \tadAxiome{sizeQueue(PriorityQueueEmpty()) = 0}
	  \tadAxiome{sizeQueue(ajouterArbre(A,f)) = sizeQueue(f) + 1}
  \end{tadAxiomes}

  \begin{tadPreconditions}{defilerArbre(f)}
    \tadPrecondition{defilerArbre(f)}{sizeQueue(f) > 0}
  \end{tadPreconditions}
\end{tad}

\subsubsection{Statistics}

\begin{tad}
    \tadNom{Statistics}
    \tadDependances{\naturel, \booleen, \entier}
    \begin{tadOperations}{getElementCount}
        \tadOperation{statistics}{}{\tadParams{Statistics}}
        \tadOperation{isEmpty}{\tadParams{Statistics}}{\tadParams{\booleen}}
        \tadOperation{contains}{\tadParams{Statistics} × \tadParams{\naturel}}{\tadParams{\booleen}}
        \tadOperation{getCount}{\tadParams{Statistics} × \tadParams{\naturel}}{\tadParams{\entier}}
        \tadOperation{incCount}{\tadParams{Statistics} × \tadParams{\naturel}}{\tadParams{Statistics}}
        \tadOperation{getCount}{\tadParams{Statistics} × \tadParams{\naturel}}{\tadParams{\entier}}
        \tadOperation{getElementCount}{\tadParams{Statistics}}{\tadParams{\entier}}
        \tadOperationAvecPreconditions{getElement}{\tadParams{Statistics} × \tadParams{\entier}}{\tadParams{\naturel}}
    \end{tadOperations}
    \begin{tadSemantiques}{getElementCount}
        \tadSemantique{statistics}{Crée une structure de statistiques vide.}
        \tadSemantique{isEmpty}{Renvoie vrai si la structure ne contient aucun élément.}
        \tadSemantique{contains}{Renvoie vrai si l'élément donné est présent dans la structure.}
        \tadSemantique{getCount}{Renvoie le nombre d'occurences de l'élément donné.}
        \tadSemantique{incCount}{Incrémente le nombre d'occurences de l'élément donné.}
        \tadSemantique{getElementCount}{Renvoie le nombre d'éléments différents dans la structure dont le compte est non nul.}
        \tadSemantique{getElement}{Renvoie le ième élément présent. La valeur de l'indice est abstraite et peut dépendre de l'implémentation.}
    \end{tadSemantiques}
    \begin{tadAxiomes}
        \tadAxiome{contains(statistics, element) \Leftrightarrow getCount(statistics, element) > 0}
        \tadAxiome{getElementCount(statistics) = 0 \Leftrightarrow isEmpty(statistics)}
        \tadAxiome{getElementCount(incCount(stats, el), el) = getElementCount(stats, el) + 1}
    \end{tadAxiomes}
    \begin{tadPreconditions}{getElement(statistics, i)}
        \tadPrecondition{getElement(statistics, i)}{i<getElementCount(statistics)}
    \end{tadPreconditions}
\end{tad}


\subsection{Analyse Descendante}

\begin{figure}[!h]
	\small
	\centering
	\begin{analyseDescendante}
		\boite{compress}{FichierBinaire}{FichierBinaire}{0}{0}
		\boite{computeStatistics}{FichierBinaire}{Statistics}{-5}{-1}
		\boite{buildHuffmanTree}{Statistics}{HuffmanTree}{-2}{-3}
		\boite{createPriorityQueue}{Statistics}{PriorityQueue}{-2}{-5}
		\boite{codingTable}{HuffmanTree}{CodingTable}{4}{-4}
		\boite{encode}{CodingTable}{FichierBinaire}{4}{-1}

		\utilise{compress}{computeStatistics}
		\utilise{compress}{buildHuffmanTree}
		\utilise{buildHuffmanTree}{createPriorityQueue}
		\utilise{compress}{codingTable}
		\utilise{compress}{encode}
	\end{analyseDescendante}
	\caption{Analyse descendante de compress}
\end{figure}


\begin{figure}[!h]
	\small
	\centering
	\begin{analyseDescendante}
		\boite{decompress}{FichierBinaire}{FichierBinaire}{0}{0}
		\boite{readStatistics}{FichierBinaire}{Statistics}{-5}{-2}
		\boite{buildHuffmanTree}{Statistics}{HuffmanTree}{0}{-3}
		\boite{createPriorityQueue}{Statistics}{PriorityQueue}{0}{-5}
		\boite{decode}{HuffmanTree}{FichierBinaire}{4}{-2}

		\utilise{decompress}{readStatistics}
		\utilise{decompress}{buildHuffmanTree}
		\utilise{buildHuffmanTree}{createPriorityQueue}
		\utilise{decompress}{decode}
	\end{analyseDescendante}
	\caption{Analyse descendante de decompress}
\end{figure}


\newpage
\section{Conception}

\subsection{Conception Préliminaire}
La phase de conception préliminaire dans le cadre de notre projet représente le point de départ de la transformation de nos idées initiales en un plan d'action concret. Chaque membre de notre équipe s'est vu attribuer une responsabilité spécifique lors de cette étape cruciale du cycle en V. L'objectif était clair : définir les grandes lignes de notre compresseur en écrivant les signatures des fonctions et procédures en pseudo-code. Cette approche méthodique visait à jeter les bases solides nécessaires pour une implémentation réussie ultérieure en langage C.\newline
\begin{algorithme} \signatureFonction{createBinaryCode}
    {}{PriorityQueue}{} \end{algorithme}

\subsubsection{Byte}

\begin{algorithme}
    \signatureFonction{octet}{\paramEntree{b1, b2, b3, b4, b5, b6, b7, b8: Bit}}{Byte}{}
    \signatureFonction{getBit}{\paramEntree{byte: Byte}, \paramEntree{i: \naturel}}{Bit}{bit < 8}
    \signatureProcedure{setBit}{\paramEntreeSortie{byte: Byte}, \paramEntree{i: \naturel}, \paramEntree{bit: Bit}}{bit < 8}
    \signatureFonction{byteToNatural}{\paramEntree{byte: Byte}}{\naturel}{}
\end{algorithme}

\subsubsection{CodingTable}

\begin{algorithme}
    \signatureFonction{codingTable}{tree: HuffmanTree}{table: CodingTable}{}

    \signatureFonction{isEmpty}{table: CodingTable}{\booleen}{}

    \signatureFonction{contains}{table: CodingTable, byte: Byte}{\booleen}{}
    
    \signatureProcedure{add}{\paramEntreeSortie{table: CodingTable} \paramEntree{byte: Byte, code: BinaryCode}}{}
    
    \signatureFonction{getBinaryCode}{table: CodingTable, b: Byte}{BinaryCode}{char $\in$ table}
    
    \signatureFonction{getByte}{table: CodingTable, code: Binary}{Byte}{code $\in$ table}
\end{algorithme}

\subsubsection{HuffmanTree}

\begin{algorithme}
    \signatureFonction{getRightChild}{ht: \typePointeur{HuffmanTreeNode}}{\typePointeur{HuffmanTreeNode}}{non isALeaf(ht)}{}
\end{algorithme}

\begin{algorithme}
    \signatureFonction{getLeftChild}{ht: \typePointeur{HuffmanTreeNode}}{\typePointeur{HuffmanTreeNode}}{non isALeaf(ht)}{}
\end{algorithme}

\begin{algorithme}
    \signatureFonction{getByte}{ht : HuffmanTree}{Byte}{isALeaf(ht)}
\end{algorithme}

\begin{algorithme}
    \signatureFonction{getOccurence}{ht : HuffmanTree}{\entier}{}
\end{algorithme}

\begin{algorithme}
    \signatureFonction{isALeaf}{ht : HuffmanTree}{\booleen}{}
\end{algorithme}
\begin{algorithme} \signatureFonction{PriorityQueueEmpty}
    {}{PriorityQueue}{} \end{algorithme}

\subsubsection{Statistics}

\begin{algorithme} 
    \signatureFonction{statistics}{}{Statistics}{} 
\end{algorithme}

\begin{algorithme} 
    \signatureFonction{isEmpty}{stats: Statistics}{\booleen}{} 
\end{algorithme}

\begin{algorithme} 
    \signatureFonction{contains}{stats: Statistics, e: Element}{\booleen}{} 
\end{algorithme}

\begin{algorithme} 
    \signatureFonction{getCount}{stats: Statistics, e: Element}{\entier}{} 
\end{algorithme}

\begin{algorithme}
    \signatureProcedure{incCount}{\paramEntree{e: Element} \paramEntreeSortie{stats: Statistics}}{}
\end{algorithme}

\begin{algorithme}
    \signatureFonction{getElementCount}{stats: Statistics}{\entier}{}
\end{algorithme}

\begin{algorithme}
    \signatureFonction{getElement}{stats: Statistics, i: \entier}{Element}
    {getElement(statistics, i): i<getElementCount(statistics)}
\end{algorithme}
\subsubsection{Compress}

\begin{algorithme}
    \signatureProcedure{compress}{\paramEntreeSortie{f : FichierBinaire}}{}
\end{algorithme}

\begin{algorithme}
    \signatureFonction{computeStatistics}{f : FichierBinaire}{Statistics}{}
\end{algorithme}

\begin{algorithme}
    \signatureFonction{buildHuffmanTree}{stats : Statistics}{HuffmanTree}{}
\end{algorithme}

\begin{algorithme}
    \signatureFonction{codingTable}{ht : HuffmanTree}{CodingTable}{}
\end{algorithme}

\begin{algorithme}
    \signatureProcedure{browseTree}{\paramEntreeSortie{ht : HuffmanTree, table : CodingTable}, \paramEntree{bc : BinaryCode}}{}
\end{algorithme}

\begin{algorithme}
    \signatureProcedure{browseTree}{\paramEntreeSortie{f : FichierBinaire}, \paramEntree{stats :Statistics}}{}
\end{algorithme}



\subsection{Conception Détaillée}
La phase de conception détaillée marque une étape cruciale dans notre projet. Il s'agit de définir, en pseudo-code, de manière exhaustive les étapes algorithmiques, fournissant ainsi une feuille de route claire pour la mise en œuvre pratique. Cette étape de notre projet de compresseur Huffman représente une transition cruciale vers la matérialisation des idées abstraites en un ensemble d'instructions détaillées, 
\subsubsection{BinaryCode}

\begin{algorithme}
    \begin{enregistrement}{BinaryCode}
        \champEnregistrement{bits}{\tableauUneDimension{1..MAX}{de}{ Bit}}
        \champEnregistrement{length}{\naturel}
    \end{enregistrement}
\end{algorithme}

\begin{algorithme}
    \typeEnumere{Bit}{ZERO, ONE}
\end{algorithme}

\begin{algorithme}
    \fonction{binaryCode}{}{BinaryCode}{}{bc : BinaryCode}{
        \affecter{bc.length}{0}
        \retourner{bc}
    }
\end{algorithme}

\begin{algorithme}
    \procedure{addBit}{\paramEntreeSortie{bc : BinaryCode}, \paramEntree{b : Bit}}{}{}{
        \affecter{bc.bits[bc.length]}{b}
        \affecter{bc.length}{bc.length + 1}
    }
\end{algorithme}

\begin{algorithme}
    \fonction{getLength}{bc : BinaryCode}{\naturel}{}{}{
        \retourner{bc.length}}

\end{algorithme}

\begin{algorithme}
    \fonction{getBit}{bc : BinaryCode, position : \naturel}{Bit}{0 <= position < getLength(bc)}{}{
        \retourner{bc.bits[position]}
    }
\end{algorithme}

\begin{algorithme}
    \procedure{concatenate}{\paramEntreeSortie{bc1 : BinaryCode}, \paramEntree{bc2 : BinaryCode}}{}{i : \naturel}{
        \pour{i}{1}{getLength(bc2)}{}{
            \instruction{addBit(bc1, getBit(bc2, i))}
        }
    }
\end{algorithme}
\subsubsection{Byte}

\begin{algorithme}
    \type{Byte}{\tableau{1..8}{de}{Bit}}
\end{algorithme}
\vspace*{0.5cm}

\begin{algorithme}
    \fonction{byte}{b1,b2,b3,b4,b5,b6,b7,b8:Bit}{Byte}{}
    {res : Byte}
    {\affecter{res}{[ ]}
    \affecter{res[0]}{b1}
    \affecter{res[1]}{b2}
    \affecter{res[2]}{b3}
    \affecter{res[3]}{b4}
    \affecter{res[4]}{b5}
    \affecter{res[5]}{b6}
    \affecter{res[6]}{b7}
    \affecter{res[7]}{b8}
    \retourner{res}}
\end{algorithme}
\vspace*{0.5cm}

\begin{algorithme}
    \fonction{byteToNatural}{byte: Byte}{\naturel}{}{i,res: \naturel}
    {\affecter{res}{0}
    \pour{i}{0}{7}{}{
        \affecter{res}{res + byte[i]*(2**(7-i))}}
    \retourner{res}}
\end{algorithme}
\vspace*{0.5cm}

\begin{algorithme}
    \fonction{byteFromNatural}{natural: \naturel}{Byte}{}
    {res : Byte}
    {\affecter{res}{[ ]}
    \pour{i}{0}{7}{}{
        \affecter{res[7-i]}{natural \% 2}
        \affecter{natural}{{natural / 2}}
    }
    \retourner{res}}
\end{algorithme}

\begin{algorithme}
    \fonction{byteEquals}{byte1, byte2: Byte}{\booleen}{}
    {res : \booleen}
    {\affecter{res}{1}
    \affecter{i}{0}
    \tantque{(res and i<8)}{
        \sialorssinon{byte1[i] != byte2[i]}
        {\affecter{res}{0}}{}
        \affecter{i}{i+1}
    }
    \retourner{res}}
\end{algorithme}

\subsubsection{CodingTable}

\constante{MAX}{256}

\begin{algorithme}
    \begin{enregistrement}{optionalBinaryCode}
        \champEnregistrement{present}{\booleen}
        \champEnregistrement{binaryCode}{BinaryCode}
    \end{enregistrement}
\end{algorithme}
\vspace*{0.5cm}

\begin{algorithme}
    \begin{enregistrement}{CodingTable}
        \champEnregistrement{tab}{\tableau{1..MAX}{de}{optionalBinaryCode}}
    \end{enregistrement}
\end{algorithme}
\vspace*{0.5cm}

\begin{algorithme}
    \fonction{codingTable}{}{CodingTable}{}{table : CodingTable,i : \naturelNonNul}
    {\pour{i}{1}{MAX}{}{
        \instruction{\affecter{\champ{table}{tab[i].present}}{0}}
    }
    \retourner{table}}
\end{algorithme}
\vspace*{0.5cm}

\begin{algorithme}
    \fonction{isEmpty}{table : CodingTable}{\booleen}{}{i : \naturelNonNul}
    {\pour{i}{1}{MAX}{}{
        \sialorssinon{\champ{table}{tab[i].present}}
        {\retourner{0}}{}
    }
    {\retourner{1}}}
\end{algorithme}
\vspace*{0.5cm}

\begin{algorithme}
    \fonction{contains}{table : CodingTable, byte : \naturel}{\booleen}{}{}
    {\retourner{\champ{table}{tab[byte].present}}}
\end{algorithme}
\vspace*{0.5cm}

\begin{algorithme}
    \fonction{getBinaryCode}{table : CodingTable, byte : \naturel}{BinaryCode}{contains(table,byte)}{}
    {\retourner{\champ{table}{tab[byte]}{.binaryCode}}}
\end{algorithme}
\vspace*{0.5cm}

\begin{algorithme}
    \procedure{add}{\paramEntreeSortie{table: CodingTable} \paramEntree{byte: \naturel, code : BinaryCode} }{non contains(table,byte)}{}
    {\affecter{\champ{table}{tab[byte]}{.present}}{VRAI}
    \affecter{\champ{table}{tab[byte]}{.binaryCode}}{code}}
\end{algorithme}
\vspace*{0.5cm}

\begin{algorithme}
    \fonction{getByte}{table : CodingTable, code : BinaryCode}{Byte}{}{i : \naturelNonNul}
    {\pour{i}{1}{MAX}{}{
        \sialorssinon{\champ{table}{tab[i].present} et \champ{table}{tab[i].binaryCode}=code}
        {\retourner{byte(i et 0b80, i et 0b40, i et 0b20,i et 0b10,i et 0b08,i et 0b04,i et 0b02,i et 0b01)}}{}
    }
    {\retourner{byte(0,0,0,0,0,0,0,0)}}}
\end{algorithme}

\subsubsection{HuffmanTree}

\begin{algorithme}
    \type{HuffmanTree}{\typePointeur{HuffmanTreeNode}}
    \vspace*{0.5cm}

    \begin{enregistrement}{HuffmanTreeNode}
        \champEnregistrement{occurence}{\entier}
        \champEnregistrement{leftChild}{\typePointeur{HuffmanTreeNode}}
        \champEnregistrement{rightChild}{\typePointeur{HuffmanTreeNode}}
    \end{enregistrement}
    \vspace*{0.5cm}

    \fonction{getRightChild}{ht: \typePointeur{HuffmanTreeNode}}{\typePointeur{HuffmanTreeNode}}{!isALeaf(ht)}{}{
        \retourner{ht.rightChild}
    }
    \vspace*{0.5cm}

    \fonction{getLeftChild}{ht: \typePointeur{HuffmanTreeNode}}{\typePointeur{HuffmanTreeNode}}{!isALeaf(ht)}{}{
        \retourner{ht.leftChild}
    }
    \vspace*{0.5cm}

    \fonction{getByte}{ht: \typePointeur{HuffmanTreeNode}}{Byte}{isALeaf(ht)}{}{
        \retourner{ht.occurence}
    }
    \vspace*{0.5cm}

    \fonction{getOccurence}{ht: \typePointeur{HuffmanTreeNode}}{\entier}{}{}{
        \retourner{ht.occurence}
    }
    \vspace*{0.5cm}

    \fonction{isALeaf}{ht: \typePointeur{HuffmanTreeNode}}{\booleen}{}{}{
        \retourner{ht.leftChild = NIL and ht.rightChild = NIL}
    }
    \vspace*{0.5cm}

    \fonction{createLeaf}{occurence: \entier}{\typePointeur{HuffmanTreeNode}}{}{}{
        \affecter{node}{\allouer{HuffmanTreeNode}}
        \affecter{node.occurence}{occurence}
        \affecter{node.leftChild}{NIL}
        \affecter{node.rightChild}{NIL}
        \retourner{node}
    }
    \vspace*{0.5cm}

    \fonction{creerNoeud}{leftChild, rightChild: \typePointeur{HuffmanTreeNode}}{\typePointeur{HuffmanTreeNode}}{(leftChild $\neq$ NIL) ET (rightChild $\neq$ NIL)}{}{
        \affecter{node}{\allouer{HuffmanTreeNode}}
        \affecter{node.occurence}{leftChild.occurence + rightChild.occurence}
        \affecter{node.leftChild}{leftChild}
        \affecter{node.rightChild}{rightChild}
        \retourner{node}
    }
    \vspace*{0.5cm}

    \procedure{destroy}{ht: \typePointeur{HuffmanTreeNode}}{}{}{
        \sialorssinon{ht $\neq$ NIL}
        {\instruction{destroy(ht.leftChild)}
        \instruction{destroy(ht.rightChild)}
        \instruction{free(ht)}}
        {}
    }
    \vspace*{0.5cm}

    \fonction{createTree}{root: \typePointeur{HuffmanTreeNode}}{HuffmanTree}{root $\neq$ NIL}{}{
        \retourner{root}
    }
    \vspace*{0.5cm}
\end{algorithme}

\subsubsection{CodingTable}

\constante{MAX}{100}
\begin{enregistrement}
    \champEnregistrement{elements}{\tableau{1..MAX}{de}{HuffmanTree}}
    \champEnregistrement{nbElements}{\naturelNonNul}
\end{enregistrement}

\begin{algorithme} 
    \signatureFonction{codingTable}{tree: HuffmanTree}{table: CodingTable}{} 
\end{algorithme}

\begin{algorithme}
    \signatureFonction{getBinaryCode}{table: CodingTable, char: \caractere}{Binary}{char $\in$ table}
\end{algorithme}

\begin{algorithme}
    \signatureFonction{getChar}{table: CodingTable, code: Binary}{\caractere}{code $\in$ table}
\end{algorithme}

\begin{algorithme}
    \signatureFonction{tableSize}{table: CodingTable}{\naturelNonNul}{}
\end{algorithme}
\subsubsection{Statistics}

\constante{MAX}{256}

\begin{algorithme}
    \begin{enregistrement}{Statistics}
        \champEnregistrement{element}{\tableauUneDimension{1..MAX}{de}{\naturel}}
        \champEnregistrement{length}{\naturel}
    \end{enregistrement}
\end{algorithme}

\begin{algorithme}
    \fonction{statistics}{}{Statistics}{}{s : Statistics, i : \naturelNonNul}{
        \pour{i}{1}{MAX}{}{
            \instruction{\affecter{s[i]}{0}}
        }
        \retourner{s}}
\end{algorithme}

\begin{algorithme}
    \fonction{contains}{stats : Statistics, b : Byte}{\booleen}{}{}{
        \retourner{getCount{stats, i} > 0}}
\end{algorithme}

\begin{algorithme}
    \fonction{getCount}{stats : Statistics, i : \naturel}{\naturel}{}{}{
        \retourner{stats[i]}
    }
\end{algorithme}

\begin{algorithme}
    \procedure{incCount}{\paramEntree{i : \naturel} , \paramEntreeSortie{stats : Statistics}}{}{}{
        \affecter{stats[i]}{stats[i] + 1}
    }
\end{algorithme}
\subsubsection{compress}

\begin{algorithme}
    \fonction{computeStatistics}{file: FichierBinaire}{Statistics}{}{
            {stats: Statistics;}
            {byte: Byte;}
            {inputChar: unsigned char;}
    }{
        \affecter{stats}{statistics()}
        \tantque{fread(inputChar, 1, 1, file) == 1}{
            \affecter{byte}{fromNatural((unsigned int) inputChar)}
            \instruction{incCount(stats, byte)}
        }
        \retourner{stats}
    }
    \vspace*{0.5cm}

    \procedure{saveStatistics}{stats: Statistics, file: FichierBinaire}{}{
            {i: \naturel;}
            {count: \naturel;}
    }{
        \pour{i}{0}{MAX-1}{}{
            \affecter{count}{getCount(stats, fromNatural(i))}
            \instruction{writeVarInt(file, count)}
        }
    }
    \vspace*{0.5cm}

    \procedure{streamCompress}{sourceFile: FichierBinaire, outputFile: FichierBinaire, codingTable: CodingTable}{}{
            {inputChar: unsigned char;}
            {buffer: BinaryCode;}
            {newBits: BinaryCode;}
    }{
        \affecter{buffer}{binaryCode()}
        \tantque{fread(inputChar, 1, 1, sourceFile) == 1}{
            \affecter{newBits}{getBinaryCode(codingTable, fromNatural((unsigned int) inputChar))}
            \instruction{concatenate(buffer, newBits)}
            \tantque{getLength(buffer) >= 8}{
                \affecter{inputChar}{byteToNatural(removeFirstByte(buffer))}
                \instruction{fwrite(inputChar, sizeof(char), 1, outputFile)}
            }
        }
        \sialorssinon{getLength(buffer) > 0}{
            \tantque{getLength(buffer) < 8}{
                \instruction{addBit(buffer, 0)}
            }
            \affecter{inputChar}{byteToNatural(removeFirstByte(buffer))}
            \instruction{fwrite(inputChar, sizeof(char), 1, outputFile)}
        }{}
    }
    \vspace*{0.5cm}

    \procedure{compress}{sourceFile: FichierBinaire, outputFile: FichierBinaire}{}{
            {statistics: Statistics;}
            {pq: PriorityQueue;}
            {ht: HuffmanTree;}
            {ct: CodingTable;}
    }{
        \affecter{statistics}{computeStatistics(sourceFile)}
        \affecter{pq}{fromStatistics(statistics)}
        \affecter{ht}{intoHuffmanTree(pq)}
        \affecter{ct}{fromHuffmanTree(ht)}
        \instruction{writeString(outputFile, "HUFFMAN")}
        \instruction{saveStatistics(statistics, outputFile)}
        \instruction{streamCompress(sourceFile, outputFile, ct)}

        \instruction{close(sourceFile)}
        \instruction{close(outputFile)}
        \instruction{destroy(ht)}
    }
\end{algorithme}

\subsubsection{decompress}

\begin{algorithme}
    \procedure{decompress}{\paramEntreeSortie{f: FichierBinaire}}{}{
        {stats: Statistics;}
        {ht: HuffmanTree;}
    }
    {\affecter{stats}{readStatistics}
    \affecter{ht}{buildHuffmanTree(stats)}
    \affecter{f}{decode(ht)}
    }
\end{algorithme}
\vspace*{0.5cm}

\section{Implémentation}
Lors de la phase de développement, les membres ont mis en œuvre la partie du projet qui leur avait été attribuée. Ce passage fluide entre la conception et la programmation a permis d'exploiter pleinement les compétences de chaque étudiant du groupe.
\subsection{Developpement}

\lstdefinestyle{customc}{
    belowcaptionskip=1\baselineskip,
    breaklines=true,
    frame=L,
    xleftmargin=\parindent,
    language=C,
    showstringspaces=false,
    basicstyle=\footnotesize\ttfamily,
    keywordstyle=\bfseries\color{green!40!black},
    commentstyle=\itshape\color{purple!40!black},
    identifierstyle=\color{blue},
    stringstyle=\color{orange},
    captionpos=t,
    caption=\lstname
}


\lstset{style=customc}

\subsubsection{.h}

\lstinputlisting{../programme/include/binaryCode.h}
\lstinputlisting{../programme/include/byte.h}
\lstinputlisting{../programme/include/codingTable.h}
\lstinputlisting{../programme/include/huffmanTree.h}
\lstinputlisting{../programme/include/priorityQueue.h}
\lstinputlisting{../programme/include/statistics.h}

\subsubsection{.c}

\lstinputlisting{../programme/src/binaryCode.c}
%\lstinputlisting{../programme/src/byte.c}
\lstinputlisting{../programme/src/codingTable.c}
\lstinputlisting{../programme/src/huffmanTree.c}
\lstinputlisting{../programme/src/priorityQueue.c}
\lstinputlisting{../programme/src/statistics.c}
\lstinputlisting{../programme/src/compress.c}
%\lstinputlisting{../programme/src/decompress.c}

\subsection{Tests unitaires}

\lstinputlisting{../programme/src/tests/testBinaryCode.c}
\lstinputlisting{../programme/src/tests/testByte.c}
\lstinputlisting{../programme/src/tests/testCodingTable.c}
\lstinputlisting{../programme/src/tests/testHuffmanTree.c}
\lstinputlisting{../programme/src/tests/testPriorityQueue.c}
\lstinputlisting{../programme/src/tests/testStatistics.c}

\section{Conclusion}
La conclusion de notre rapport met en lumière l'impact positif de notre approche structurée dans le contexte du cycle en V. En attribuant à chaque membre une partie spécifique à chaque étape, nous avons créé une dynamique d'équipe.Cette combinaison a abouti à une expertise approfondie dans chaque aspect du compresseur Huffman tout en favorisant l'entraide et la compréhension globale du projet.LA répartition des tâches  a non seulement optimisé la réalisation du compresseur, mais a également renforcé les compétences individuelles de chaque membre, contribuant ainsi à une expérience d'apprentissage enrichissante pour l'ensemble de l'équipe.

\subsection{Matrice de répartition des tâches}

\begin{table}[]
\centering
\resizebox{\textwidth}{!}{%
\begin{tabular}{|l|l|l|l|l|l|l|}
\hline
 &  & B. Quentin & B. Angelo & G. Simon & L. Houssam & T. Florent \\ \hline
\multirow{6}{*}{TADs} & HuffmanTree & X &  &  &  &  \\ \cline{2-7}
 & Byte &  & X &  &  &  \\ \cline{2-7} 
 & Statistics &  &  & X &  &  \\ \cline{2-7} 
 & BinaryCode &  &  &  & X &  \\ \cline{2-7} 
 & PriorityQueue &  &  &  &  & X \\ \cline{2-7} 
 & CodingTable &  &  &  &  & X \\ \hline
\multirow{8}{*}{Conception préliminaire} & HuffmanTree &  & X &  &  &  \\ \cline{2-7} 
 & Byte &  &  & X &  &  \\ \cline{2-7} 
 & Statistics &  &  &  & X &  \\ \cline{2-7} 
 & BinaryCode &  &  &  &  & X \\ \cline{2-7} 
 & PriorityQueue & X &  &  &  &  \\ \cline{2-7} 
 & CodingTable & X &  &  &  &  \\ \cline{2-7} 
 & Compress &  & X &  &  &  \\ \cline{2-7} 
 & Decompress &  &  & X &  &  \\ \hline
\multirow{8}{*}{Conception détaillée} & HuffmanTree &  &  & X &  &  \\ \cline{2-7} 
 & Byte &  &  &  & X &  \\ \cline{2-7} 
 & Statistics &  &  &  &  & X \\ \cline{2-7} 
 & BinaryCode & X &  &  &  &  \\ \cline{2-7} 
 & PriorityQueue &  & X &  &  &  \\ \cline{2-7} 
 & CodingTable &  & X &  &  &  \\ \cline{2-7} 
 & Compress &  &  & X &  &  \\ \cline{2-7} 
 & Decompress &  &  &  & X &  \\ \hline
\multirow{8}{*}{Developpement} & HuffmanTree &  &  &  & X &  \\ \cline{2-7} 
 & Byte &  &  &  &  & X \\ \cline{2-7} 
 & Statistics & X &  &  &  &  \\ \cline{2-7} 
 & BinaryCode &  & X &  &  &  \\ \cline{2-7} 
 & PriorityQueue &  &  & X &  &  \\ \cline{2-7} 
 & CodingTable &  &  & X &  &  \\ \cline{2-7} 
 & Compress &  &  &  & X &  \\ \cline{2-7} 
 & Decompress &  &  &  &  & X \\ \hline
\multirow{8}{*}{Tests Unitaires} & HuffmanTree &  &  &  &  & X \\ \cline{2-7} 
 & Byte & X &  &  &  &  \\ \cline{2-7} 
 & Statistics &  & X &  &  &  \\ \cline{2-7} 
 & BinaryCode &  &  & X &  &  \\ \cline{2-7} 
 & PriorityQueue &  &  &  & X &  \\ \cline{2-7} 
 & CodingTable &  &  &  & X &  \\ \hline
\end{tabular}%
}
\caption{Tableau de répartiton des tâches}
\label{tab:my-table}
\end{table}
\end{document}
